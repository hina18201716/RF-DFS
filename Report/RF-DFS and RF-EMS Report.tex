\documentclass[titlepage]{article}

% Suppresses chktex warnings 
% chktex-file 1
% chktex-file 3
% chktex-file 8
% chktex-file 10
% chktex-file 17
% chktex-file 24
% chktex-file 26
% chktex-file 36
% chktex-file 44
% chktex-file 46

\usepackage{geometry}
\usepackage{tikz}
\usetikzlibrary{fit}
\usetikzlibrary{intersections}
\usepackage{pgfplots}
\usepgfplotslibrary{fillbetween}
\pgfplotsset{compat=1.18}
\pgfplotsset{hide xscale/.style={/pgfplots/xtick scale label code/.code={}}}
\pgfplotsset{hide yscale/.style={/pgfplots/ytick scale label code/.code={}}}
\usepackage{amsmath}
\usepackage{upgreek}
\usepackage{url}
\hyphenation{}	        % Correct bad hyphenation
\usepackage{graphicx}   % For including figures and pictures
\usepackage{float}      % Used to fix location of images i.e.\begin{figure}[H]
\usepackage{fancyhdr}   % For headers and footers
\pagestyle{fancy}
\usepackage{lastpage}   % Allows referencing last page (for footer)
\usepackage{titling}    % To reference title, author, and date
\usepackage{array}      % For fixed-width tables
\usepackage{makecell}   % Formats individual table cells
\usepackage{multirow}   % For multi-row columns (Top left justify header)
\usepackage[american]{circuitikz} % Used to draw circuit and block diagrams
\usepackage{appendix}   
\usepackage{colortbl}   % For coloring table cells
\definecolor{nraoblue}{rgb}{0.776,0.851,0.945}  % Color of table headers
\usepackage{enumitem}   % For formatting enumerated lists
\usepackage[parfill]{parskip}               % Removes paragraph indentation, adds line break
\usepackage[hang, flushmargin]{footmisc}    % Removes footnote indentation
\setenumerate[1]{label={(\arabic*)}}
\usepackage[backend=biber,style=numeric,sorting=none]{biblatex}
\addbibresource{references.bib}
\usepackage{hyperref}   % Allows hyperlinks
\hypersetup{
  colorlinks   = true,  % Colors links instead of ugly boxes
  filecolor    = blue,  % Color for local hyperlinks
  linkcolor    = ., % Color for internal hyperlinks
  citecolor    = ., % Color for citations
  urlcolor     = blue,  % Color for linked urls
}
\usepackage{caption}    % Changes figure/table caption color and size
\definecolor{captioncolor}{RGB}{79,129,189}
\captionsetup{font={
    small, 
    color=captioncolor,
    bf,
}}
 

% Changes font from Computer Modern to Gill Sans MT
% This requires XeLaTeX or LuaTeX to function
% Add "latex-workshop.latex.recipe.default": "latexmk (xelatex)" to workspace.json
\usepackage{fontspec}
\setmainfont{gil.ttf}[
    BoldFont        = gil-b.ttf,
    ItalicFont      = gil-i.ttf,
    BoldItalicFont  = gil-bi.ttf,
]

% Global TikZ parameters
\tikzset{every picture/.style={/utils/exec={\fontfamily{lmr}}}}
\ctikzset{sources/fill=red!20}
\ctikzset{chips/fill=cyan!20}
\ctikzset{electromechanicals/fill=blue!20}
\ctikzset{blocks/fill=green!20}
\ctikzset{amplifiers/fill=yellow!30}
\tikzset{amp/.append style={fill=yellow!30}}
\tikzset{twoport/.append style={fill=cyan!20}}
% TikZ node[server] symbol
\makeatletter
\pgfkeys{/pgf/.cd,
  parallelepiped offset x/.initial=2mm,
  parallelepiped offset y/.initial=2mm
}
\pgfdeclareshape{parallelepiped}
{
  \inheritsavedanchors[from=rectangle] % this is nearly a rectangle
  \inheritanchorborder[from=rectangle]
  \inheritanchor[from=rectangle]{north}
  \inheritanchor[from=rectangle]{north west}
  \inheritanchor[from=rectangle]{north east}
  \inheritanchor[from=rectangle]{center}
  \inheritanchor[from=rectangle]{west}
  \inheritanchor[from=rectangle]{east}
  \inheritanchor[from=rectangle]{mid}
  \inheritanchor[from=rectangle]{mid west}
  \inheritanchor[from=rectangle]{mid east}
  \inheritanchor[from=rectangle]{base}
  \inheritanchor[from=rectangle]{base west}
  \inheritanchor[from=rectangle]{base east}
  \inheritanchor[from=rectangle]{south}
  \inheritanchor[from=rectangle]{south west}
  \inheritanchor[from=rectangle]{south east}
  \backgroundpath{
    % store lower right in xa/ya and upper right in xb/yb
    \southwest \pgf@xa=\pgf@x \pgf@ya=\pgf@y
    \northeast \pgf@xb=\pgf@x \pgf@yb=\pgf@y
    \pgfmathsetlength\pgfutil@tempdima{\pgfkeysvalueof{/pgf/parallelepiped
      offset x}}
    \pgfmathsetlength\pgfutil@tempdimb{\pgfkeysvalueof{/pgf/parallelepiped
      offset y}}
    \def\ppd@offset{\pgfpoint{\pgfutil@tempdima}{\pgfutil@tempdimb}}
    \pgfpathmoveto{\pgfqpoint{\pgf@xa}{\pgf@ya}}
    \pgfpathlineto{\pgfqpoint{\pgf@xb}{\pgf@ya}}
    \pgfpathlineto{\pgfqpoint{\pgf@xb}{\pgf@yb}}
    \pgfpathlineto{\pgfqpoint{\pgf@xa}{\pgf@yb}}
    \pgfpathclose
    \pgfpathmoveto{\pgfqpoint{\pgf@xb}{\pgf@ya}}
    \pgfpathlineto{\pgfpointadd{\pgfpoint{\pgf@xb}{\pgf@ya}}{\ppd@offset}}
    \pgfpathlineto{\pgfpointadd{\pgfpoint{\pgf@xb}{\pgf@yb}}{\ppd@offset}}
    \pgfpathlineto{\pgfpointadd{\pgfpoint{\pgf@xa}{\pgf@yb}}{\ppd@offset}}
    \pgfpathlineto{\pgfqpoint{\pgf@xa}{\pgf@yb}}
    \pgfpathmoveto{\pgfqpoint{\pgf@xb}{\pgf@yb}}
    \pgfpathlineto{\pgfpointadd{\pgfpoint{\pgf@xb}{\pgf@yb}}{\ppd@offset}}
  }
}
\makeatother

\tikzset{
  ports/.style={
    line width=0.3pt,
    top color=gray!20,
    bottom color=gray!80
  },
  server/.style={
    parallelepiped,
    fill=white, draw,
    minimum width=0.35cm,
    minimum height=0.75cm,
    parallelepiped offset x=3mm,
    parallelepiped offset y=2mm,
    xscale=-1,
    path picture={
      \draw[top color=gray!5,bottom color=gray!40]
      (path picture bounding box.south west) rectangle 
      (path picture bounding box.north east);
      \coordinate (A-center) at ($(path picture bounding box.center)!0!(path
        picture bounding box.south)$);
      \coordinate (A-west) at ([xshift=-0.575cm]path picture bounding box.west);
      \draw[ports]([yshift=0.1cm]$(A-west)!0!(A-center)$)
        rectangle +(0.2,0.065);
      \draw[ports]([yshift=0.01cm]$(A-west)!0.085!(A-center)$)
        rectangle +(0.15,0.05);
      \fill[black]([yshift=-0.35cm]$(A-west)!-0.1!(A-center)$)
        rectangle +(0.235,0.0175);
      \fill[black]([yshift=-0.385cm]$(A-west)!-0.1!(A-center)$)
        rectangle +(0.235,0.0175);
      \fill[black]([yshift=-0.42cm]$(A-west)!-0.1!(A-center)$)
        rectangle +(0.235,0.0175);
    }  
  },
}

% The following lines define a new command \nraocite that cites references in NRAO format RD0X by redefining \cite command to remove brackets. \nraoprecite includes a prenote.
\DeclareCiteCommand{\cite} [\mkbibemph{\emph{}}]
  {\usebibmacro{prenote}}
  {\usebibmacro{citeindex}%
   \printtext[bibhyperref]{\usebibmacro{cite}}}
  {\multicitedelim}
  {\usebibmacro{postnote}}
\DeclareCiteCommand*{\cite} [\mkbibemph\emph{}]
  {\usebibmacro{prenote}}
  {\usebibmacro{citeindex}
   \printtext[bibhyperref]{\usebibmacro{citeyear}}}
  {\multicitedelim}
  {\usebibmacro{postnote}}
\newcommand{\nraocite}[1]{[RD0\cite{#1}]}
\newcommand{\nraoprecite}[2][]{[RD0\cite{#2}{, #1}]}


% ASSIGN TITLE, AUTHOR, DATE, DOCUMENT NUMBER, STATUS, HERE
\title{Title goes here}
\author{Author
    }% <-this % stops a space
\date{Jan 1, 2000}
\def\docnum{DOC NUM}
\def\status{\textcolor{red}{DRAFT}}

% HEADERS AND FOOTERS
\renewcommand{\headrulewidth}{0pt}
\fancyheadoffset[L,R]{2cm}
\fancyhead[L]{\vspace{-1cm}\includegraphics[width=2cm]{images/NRAO Logo Badge.png}}
\fancyhead[R]{
\renewcommand{\arraystretch}{1.4}
\renewcommand{\arrayrulewidth}{0.25pt}
\begin{tabular}{|w{l}{6.7cm}|w{l}{4.9cm}|w{l}{3.4cm}|}
    \hline
        \multirow[t]{2}{6.7cm}{\textit{\textbf{Title:}} \thetitle} &
        \textit{\textbf{Owner:}} \theauthor &
        \textit{\textbf{Date:}}  \thedate \\
            &   &   \\
    \hline
        \multicolumn{2}{|l|}{
            \textit{\textbf{NRAO Doc \#:}} \docnum  
        } &
        \textit{\textbf{Version:}} A \\
    \hline
\end{tabular}
}
\fancyfoot[C]{Page \textbf{\thepage} of \textbf{\pageref{LastPage}}}

 % DOCUMENT STARTS HERE
\begin{document}
\setlength{\leftmargin}{1in}        % Sets margin
\setlength{\rightmargin}{1in}       % Sets margin
\setlength{\voffset}{-1.2in}        % Moves header up
\setlength{\headheight}{3.5cm}      % Defines header height
\setlength{\textheight}{591pt}      % Extends text/body size down
\setlength{\footskip}{60pt}         % Defines gap between body and footer

\thispagestyle{fancy}
\begin{center}
     \includegraphics[width=5cm]{images/NRAO Logo Badge.png} \\
     \vspace*{0.5cm}
     \textbf{\huge\thetitle} \\
     \vspace*{0.5cm}
     \large\docnum \\
     \Large Status: \status \\
     \vspace*{0.6cm} \large
     % FIRST TABLE HERE
     \begin{tabular}{|m{6.93cm}|m{4.5cm}|m{2cm}|} \hline
        \rowcolor{nraoblue}
        \textbf{Prepared By} & \textbf{Organization} & \textbf{Date} \\ \hline
        \makecell[l]{Author\\Title} & NRAO Electronics Div. & 1/1/2000 \\ 
        \hline
    \end{tabular} \\
    \vspace*{0.8cm}
    \begin{tabular}{|m{3cm}|m{3.5cm}|m{6.93cm}|} \hline
        \rowcolor{nraoblue}
        \textbf{Approvals} & \textbf{Organization} & \textbf{Signatures} \\ \hline
    \end{tabular}
    \renewcommand{\arraystretch}{2}
    % CONTENT OF SECOND TABLE HERE
    \begin{tabular}{|w{l}{3cm}|m{3.5cm}|m{6.93cm}|} \hline
        \parbox{3cm}{\raggedright
        Name\\Title
        } & NRAO Electronics Division \raggedright &  \\ 
        \hline
        \parbox{3cm}{\raggedright
        Name\\Title
        } & NRAO Electronics Division \raggedright &  \\ 
        \hline
        \parbox{3cm}{\raggedright
        Name\\Title
        } & NRAO Electronics Division \raggedright &  \\ 
        \hline
    \end{tabular} \\
    \renewcommand{\arraystretch}{1}
    \vspace*{0.8cm}
    \begin{tabular}{|m{3cm}|m{3.5cm}|m{6.93cm}|} \hline
        \rowcolor{nraoblue}
        \textbf{Released By} & \textbf{Organization} & \textbf{Signature} \\ \hline
    \end{tabular}
    \renewcommand{\arraystretch}{1.5}
    % CONTENT OF THIRD TABLE HERE
    \begin{tabular}{|m{3cm}|m{3.5cm}|m{6.93cm}|} \hline
        \parbox{3cm}{\raggedright
        Name\\Title
        } & NRAO Electronics Division \raggedright &  \\ 
        \hline
    \end{tabular}
    \renewcommand{\arraystretch}{1}
\end{center}

% CHANGE RECORD
\newpage
\section*{Change Record}
\begin{center}
\renewcommand{\arraystretch}{1.2}
    \begin{tabular}{|m{1.5cm}|m{2.2cm}|m{2.5cm}|m{1.7cm}|m{5cm}|} \hline
        \rowcolor{nraoblue}
        Version & Date & Author & Affected\newline Section(s) & Reason\\ \hline
    \end{tabular}
\renewcommand{\arraystretch}{1.6}
    \begin{tabular}{|m{1.5cm}|m{2.2cm}|m{2.5cm}|m{1.7cm}|m{5cm}|} \hline
        01 & Aug 15, 2023 & R. Nguyen & All & Initial Draft \\ \hline
        02 & Aug 16, 2023 & \makecell[l]{T. Anderson\\R. Nguyen} & All & Edits \\ \hline
        A  & Aug 16, 2023 & T. Anderson & All & Review \\ \hline
          &          &           &     &               \\ \hline
          &          &           &     &               \\ \hline
          &          &           &     &               \\ \hline
          &          &           &     &               \\ \hline
    \end{tabular}
\renewcommand{\arraystretch}{1}
\end{center}

\newpage
\tableofcontents
\listoffigures
\thispagestyle{fancy}
\newpage

\section{Introduction}

\subsection{Purpose}


\subsection{Scope}


\subsection{Verb Convention}
“Must” for an obligation; “must not” for a prohibition. 

“May” for a discretionary action; “should” for a recommendation. 

“Will” is used to indicate a future happening/action. 


\section{Related Documents and Drawings}
\subsection{Applicable Documents}
The following documents may not be directly referenced herein, but may provide necessary context or supporting material.
\begin{center}
\begin{tabular}{|m{2cm}|m{8cm}|m{3cm}|} \hline
    \rowcolor{nraoblue}
    Ref. No. & Document Title & Rev/Doc. No.\\ \hline
    AD01 & Desiderata for Solar Observing with the EVLA & EVLAM\_70 \\ 
    \hline
    AD02 & EVLA Hardware Modifications in Support of Solar Observing & EVLAM\_72 \\
    \hline
\end{tabular}
\end{center}

\subsection{Reference Documents}
The following documents are referenced within this text:
\begin{center}
\renewcommand{\arraystretch}{1.2}
\begin{tabular}{|m{2cm}|m{8cm}|m{3cm}|} \hline
    \rowcolor{nraoblue}
    Ref. No. & Document Title & Rev/Doc. No.\\ \hline
    RD01 & \citefield{solartemp}{title} & \href{https://ipnpr.jpl.nasa.gov/progress_report/42-175/175E.pdf}{Online} \\ \hline
    RD02 & \citefield{aeh}{title} & DSOC Bookcase \\\hline
    RD03 & \citefield{tora}{title} & Privately Owned \\\hline
    RD04 & \citefield{xbandvla}{title} & \href{https://tmo.jpl.nasa.gov/progress_report/42-92/92O.PDF}{Online} \\\hline
    RD05 & \citefield{sfd1986}{title} & \href{https://www.govinfo.gov/content/pkg/GOVPUB-C13-53a55ea34f3ca8aaacf289a9caa0bee6/pdf/GOVPUB-C13-53a55ea34f3ca8aaacf289a9caa0bee6.pdf}{Online} \\\hline
\end{tabular}
\renewcommand{\arraystretch}{1}
\end{center}

\section{Section title}
\label{sec:sectionlabel}
Section text here

\subsection{Subsection}
\label{sec:subsectionlabel}
Subsection text here
\begin{enumerate}
    \item List item
    \item List item
\end{enumerate}

\subsection{Subsection}
\begin{figure}[!ht]
    \begin{center}
        \begin{circuitikz}
            \draw(0, 0)
            node[
                rxantenna, 
                xscale=-1,
                label = {[xshift=-1.4cm, yshift=-0.5cm]X-band feed}
                ]{}
            to[amp = RF Gain Block, boxed] ++ (2, 0)
            to[bandpass = Filter] ++ (2,0)
            to[detector = Rectifier] ++ (2,0)
            to[amp = DC Gain Block, boxed] ++ (3,0)
            to[qvprobe = Voltmeter] ++ (2,0)
            -| ++ (0.5,-1) node[ground]{};
        \end{circuitikz}
    \caption{Block diagram.}\label{fig:rfblock}
    \end{center}
\end{figure}    

\begin{figure}[!ht]
    \begin{center}
        \begin{circuitikz}
            \draw(0, 0) node[server, scale=1.5, name=computer]{};
            \draw(computer.south) node[anchor=north]{};
            \ctikzset{bipoles/oscope/waveform=triangle}
            \ctikzset{bipoles/oscope/width=1.0}
            \draw(computer.east)  ++ (0.6, 0) 
            coordinate(ctreast)-- ++ (1, 0)
            node[oscopeshape, anchor=west](spec){N9040b};
            \draw(spec.in 1) node[ocirc, scale=0.7]{};
            \draw(spec.in 2) node[ocirc, scale=0.7]{};
            \draw(spec.east) -- ++ (1.8, 0)
            node[twoportshape, name=s1]{} -- ++ (0.8, 0)
            to[amp, boxed, invert] ++ (1.7, 0) -- ++ (0.8, 0)
            node[twoportshape, name=s2]{};
            \draw(s1.center) node[spdt, scale=0.6]{};
            \draw(s2.center) node[spdt, xscale=-1, scale=0.6]{};
            \node[draw, rectangle, dashed, fit=(s1) (s2), inner sep=10](box){};

            \draw(s2.right up) -- ++ (1.7, 0)
            node[dinantenna, xscale=-1](ems){\ctikzflipx{EMS}};
            \draw(s2.right down) -- ++ (2.5, 0)
            node[dinantenna](dfs){DFS};

            \draw([yshift=5pt]box.north)[<-] -- ++ (0, 0.4)
            node[qfpchip, anchor=south, external pins width=0, hide numbers, name=uc, solid, scale=0.8]{$\upmu$C};
            
            
            \path[name path = border1](uc.west) -- ++ (-5, 0);
            \path[name path = line1, overlay]([yshift=18pt]ctreast) -- +(45:5);
            \draw[name intersections={of=border1 and line1}] ([yshift=18pt]ctreast) -- (intersection-1) -- (uc.west);
            \path[name path=border2](intersection-1) -- ++ (0, -5);
            \path[name path = line2, overlay]([yshift=-18pt]ctreast) -- +(-45:5);
            \draw[name intersections={of=border2 and line2}] ([yshift=-18pt]ctreast) -- (intersection-1) coordinate(p1);
            
            \draw(p1) -- ++ (2, 0)
            node[qfpchip, anchor=west, external pins width=0, hide numbers, name=acr, scale=0.8, align=center]{Motor\\Ctrl.};
            \draw([yshift=-8pt]acr.east) -- ++ (3.5, 0)
            node[elmech, rotate=90, anchor=top](m2){};
            \draw(m2) [->] -| ([yshift=-10pt]dfs.center);
            \draw([yshift=8pt]acr.east) -- ++ (1.5, 0)
            node[elmech, rotate=90, anchor=top](m1){};
            %\draw(m1) [->] -- ++ (3, 0) -- ([yshift=-10pt, xshift=-10pt]dfs.center);
            \path[name path = border3](dfs.center) -- ++ (0, -2);
            \path[name path = line3, overlay](m1.bottom) -- ++ (5, 0);
            \draw[name intersections={of=border3 and line3}] (m1.bottom) -- (intersection-1)
            node[diamondpole]{};
            \draw(m1.center) node{Az};
            \draw(m2.center) node{El};
        \end{circuitikz}
    \caption{Block diagram 2.}\label{fig:sysblock}
    \end{center}
\end{figure}
\begin{figure}[!ht]
    \begin{center}
        \begin{circuitikz}
            \draw(0, 0) node[bnc](port1){To N9040b};
            \draw(port1.hot) -- ++ (1, 0)
            node[rotary switch = 3 in 50 wiper 40, anchor=in, ](s1){};
            
            \draw(s1.out 1) -| ++ (0.75, 1.5)
            -- ++ (0.5, 0)
            to[amp, invert] ++ (1.5, 0)
            to[amp, invert] ++ (1.5, 0)
            to[amp, invert] ++ (1.5, 0) -- ++ (0.5, 0)
            coordinate(p1);
            
            \draw(s1.out 2) -- ++ (1, 0)
            to[amp, invert] ++ (1.5, 0)
            to[amp, invert] ++ (1.5, 0)
            to[amp, invert] ++ (1.5, 0) -- ++ (0.5, 0)
            coordinate(p2);

            \draw($(p1)!0.5!(p2)$) ++ (1, 0)
            node[rotary switch = 2 in 30 wiper 20, xscale=-1](s2){};
            \draw(s2.out 1) -| (p1);
            \draw(s2.out 2) -| (p2);
            \draw(s2.in) -- ++ (1, 0)
            node[bnc, xscale=-1](port2){\ctikzflipx{To EMS}};
            
            \draw(s1.out 3) -| ++ (0.75, -1.5)
            -- ++ (0.5, 0)
            to[amp, invert] ++ (1.5, 0)
            to[amp, invert] ++ (1.5, 0)
            to[amp, invert] ++ (1.5, 0) -- ++ (0.5, 0)
            -- ++ (2.4, 0)
            node[bnc, xscale=-1](port3){\ctikzflipx{To DFS}};
        \end{circuitikz}
    \caption{Block diagram 3.}\label{fig:rfampblock}
    \end{center}
\end{figure}

\section{Section title}
\label{sec:sectionlabel2}
\begin{figure}[!ht]
\begin{center}
    \begin{tikzpicture}
        \begin{axis}[
            title=$V_{out}$ vs. $P_{in}$ at 8.4 GHz,
            xlabel={$P_{in}$ (dBm)},
            ylabel={$V_{out}$ (V)},
            grid=major,
            xmin=-60, xmax=0,
            ymin=0, ymax=3,
            xtick={-60,-50,...,0},
            ytick={0,0.5,...,3},
            legend style = {
                align = left,
                cells = {anchor=east},
                legend pos=outer north east,
                }
                ]
                \path[name path=xaxis] (axis cs:-60,0) -- (axis cs:0,0);
                \addplot[smooth, blue, name path=f] table [x=Pin,y=Vout] {data/Vout vs Pin.txt};

                \addplot[opacity=0.2, green] fill between [of=f and xaxis, soft clip={domain=-47.5:-37.5}];
                \draw[dashed] (axis cs:-47.5,0) -- (-47.5, 1.25);
                \draw[dashed] (axis cs:-37.5,0) -- (-37.5, 1.25);
                \draw (axis cs:-42.5,1.25)
                node[anchor=south]{Ideal Range};

                \addplot[opacity=0.2, red] fill between [of=f and xaxis, soft clip={domain=-25:-15}];
                \draw[dashed] (axis cs:-25,0) -- (-25, 2.25);
                \draw[dashed] (axis cs:-15,0) -- (-15, 2.25);
                \draw (axis cs:-20,2.25)
                node[anchor=south]{Incorrect Range};
            \end{axis}
\end{tikzpicture}
\caption{Graph from imported data}
\label{fig:vpcurve}
\end{center}
\end{figure}

\subsection{Subsection}
Example equation below
\begin{align} \label{eq:power}
    P_{R} &= P_{sun} + P_{noise} \\
    P_{R} &= S A_e \Delta f + k_B T \Delta f \nonumber
\end{align}
\begin{align*}
    \text{where:}~S ~&\text{is the source flux density in W m$^{-2}$ Hz$^{-1}$,}\\
    A_e             ~&\text{is the effective aperture area in m$^2$}\\
    \Delta f        ~&\text{is the receiver bandwidth in Hz,}\\
    k_B             ~&\text{is the Boltzmann constant in J K$^{-1}$, and}\\
    T               ~&\text{is the system noise temperature in K.}
\end{align*}
Example table below
\begin{table}[!ht]
\centering
\begin{tabular}{c|c|c|c}
    & $S_{\min}$ & $S_{\mu}$ & $S_{\max}$ \\ \hline
    2.8 GHz & 70 SFU & 150 SFU & 280 SFU \\
    8.8 GHz & 152 SFU & 326 SFU & 608 SFU
\end{tabular}
\caption{Solar Flux Density $S$ at 8.8 GHz and 2.8 GHz} \label{tab:sfd}
\end{table}

\end{document}


